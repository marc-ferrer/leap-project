\documentclass[12pt,a4paper]{article}
\usepackage[utf8]{inputenc}
\usepackage[catalan]{babel}
\usepackage{amsmath}
\usepackage{amsfonts}
\usepackage{amssymb}
\usepackage{float}
\author{Marc Ferrer Fontirroig}
\title{Proposta de projecte fi de carrera}
\begin{document}
	\begin{center}
		\huge PROPOSTA DE PORJECTE FI DE CARRERA
	\end{center}
	\section*{Objectius}
	\thispagestyle{empty}
	L’objectiu principal d’aquest projecte fi de carrera, és realitzar una prova de concepte d’exercicis de rehabilitació de lesions de les articulacions de la mà (canell i dits) utilitzant interfícies basades en visió comercials.
	
	La utilització d’exercicis basats en videojocs, aporta un context motivacional extra que pot ajudar a augmentar la participació dels usuaris en la seva teràpia de rehabilitació, tant en temps invertit, com quant atenció prestada a aquests exercicis. Això és un aspecte fonamental per a l’èxit de la rehabilitació.
	
	La teràpia remota basada en videojocs presenta un desavantatge, i és la falta de supervisió per part del responsable de la rehabilitació, típicament un fiseoterapeuta.
	
	Per tal de suplir aquest desavantatge, un dels objectius d’aquesta prova de concepte és que el responsable de la teràpia sigui capaç de veure una reproducció dels moviments realitzats per l’usuari. D’aquesta manera es pot fer un seguiment de l’evolució de l’usuari de manera més ràpida. De la mateixa manera, el responsable de la teràpia és capaç de detectar possibles errors en la realització dels exercicis de rehabilitació, que d’una altra manera podrien tenir un efecte perjudicial per l’usuari, i aquests es poden corregir de manera gairebé immediata.
	
	En concret, es pretén utilitzar el controlador Leap Motion. Aquest, és un dispositiu capaç de detectar i capturar les dades de les mans i els dits dins el seu camp de visió, sense necessitat d'utilitzar cap tipus de marca de referència. Tot i que no està específicament dissenyat per a la rehabilitació, el seu preu reduït i les seves dimensions, en comparació a altres dispositius de visió, fan que aquest sigui un dispositiu idoni pels tipus d’exercicis proposats.
\end{document}